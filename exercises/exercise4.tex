\documentclass[./../main.tex]{subfiles}
\graphicspath{{img/}}
\begin{document}
	\color{blue}
	\begin{exercise}[Extra - Estructura fina: corrección total a la energía (valor: +1pt)]
		Considera la estructura fina del átomo de un electrón. En clase se demostró que las correcciones relativistas a la energía asociadas a la energía cinética, \(\change{E_{1}}\), y al acoplamiento spin-órbita, \(\change{E_{2}}\), están dadas por:

		\begin{equation*}
			\change{E_{1}} = -E_{n}\tfrac{Z^{2}\alpha^{2}}{n^{2}}\left[\tfrac{3}{3} - \tfrac{n}{\ell + 1/2}\right]
		\end{equation*}

		\begin{equation*}
			\change{E_{2}} = -E_{n}\dfrac{(Z\alpha)^{2}}{2n\ell(\ell + 1/2)(\ell + 1)} \mul
			\begin{cases*}
				0 & si \(\ell = 0\)\\
				\ell & si \(j = \ell + 1/2\)\\
				-(\ell + 1) & si \(j = \ell - 1/2\)
			\end{cases*}
		\end{equation*}

		Mientras que la corrección asociada al término de Darwin, está dada por \(\change{E_{3}}\) encontrado en el problema 3. Demuestra que la corrección total \(\change{E_{T}} = \change{E_{1}} + \change{E_{2}} + \change{E_{3}}\) está dada por

		\begin{equation*}
			\change{E_{T}} = E_{n}\tfrac{Z^{2}\alpha^{2}}{n}\left[\tfrac{1}{j + 1/2} + \tfrac{3}{4n}\right].
		\end{equation*}
	\end{exercise}
\end{document}
