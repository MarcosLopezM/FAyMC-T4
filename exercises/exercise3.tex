\documentclass[./../main.tex]{subfiles}
\graphicspath{{img/}}

\begin{document}
    \begin{exercise}[Efecto Stark]
        Cuando un átomo se encuentra en presencia de un campo eléctrico constante \(\vect{E}_{s}\) los niveles de energía se desplazan, un fenómeno conocido como ``\textbf{Efecto Stark}''. En este problema analizamos únicamente el efecto Stark en los estados \(n = 1\) y \(n = 2\) del átomo de hidrógeno. Supongamos que el campo eléctrico apunta en la dirección \(z\), de manera que la energía potencial del electrón es

        \begin{equation*}
            \op{W}{S} = -\e E_{s}z = \e E_{s}r\cos\theta.
        \end{equation*}

        Considera que \(\op{W}{S}\) es una perturbación sobre el átomo de hidrógeno de Bohr (es decir, no es necesario que consideres la estructura fina e hiperfina). También puedes ignorar al spin del electrón en este problema.

        \begin{enumerate}
            \item (Valor: 1pt) - Demuestra que, a primer orden, esta perturbación no altera al estado fundamental.
            
            \begin{solution}
                Sabemos que el estado base es no-degenerado, pues 
                \(n = 1\),

                \begin{equation*}
                    0 \leq \ell < 1 \implies \ell = 0 \implies m = 0.
                \end{equation*}

                Recordando que la corrección a primer orden para la energía del estado base está dada como

                \begin{align*}
                    E_{0}^{(1)} &= \dmatrixel{100}{\op{W}{S}},\\
                    &= \dmatrixel{100}{-\e E_{S}r\cos\theta},\\
                    E_{0}^{(1)} &= -\e E_{S}\dmatrixel{100}{r\cos\theta},
                \end{align*}

                donde \(\ket{100} = \psi_{100} = \tfrac{1}{\sqrt{\pi a_{\mu}^{3}}}\e^{-r/a_{\mu}}\).

                Así,

                \begin{align*}
                    E_{0}^{(1)} &= -\e E_{S}\int\left(\dfrac{1}{\sqrt{\pi a_{\mu}^{3}}}\e^{-r/a_{\mu}}\right)(r\cos\theta)\left(\dfrac{1}{\sqrt{\pi a_{\mu}^{3}}}\e^{-r/a_{\mu}}\right)r^{2}\sin\theta\odif{r,\theta,\phi},\\
                    &= \dfrac{-\e E_{S}}{\pi a_{\mu}^{3}}\int_{0}^{\infty}r^{3}\e^{-2r/a_{\mu}}\odif{r}\int_{0}^{\pi}\cos\theta\sin\theta\odif{\theta}\int_{0}^{2\pi}\odif{\phi}.
                \end{align*}

                Por lo que la perturbación a primer orden no altera al estado fundamental, \idest

                \begin{empheq}[box = \color{pinkwave}\fbox]{equation*}
                    E_{0}^{(1)} = 0.
                \end{empheq}
            \end{solution}
            
            \item (Valor: 3pt) - El primer estado excitado presenta degeneración cuádruple: \(\psi_{200}\), \(\psi_{211}\), \(\psi_{210}\) y \(\psi_{21-1}\). Usando teoría de perturbaciones para el caso degenerado, determina la corrección de la energía a primer orden. ¿En cuántos niveles se desdobla la energía \(E_{2}\)?
            
            \begin{solution}
                Por teoría de perturbaciones par ael caso degenerado sabemos que

                \begin{equation*}
                    W_{ij} = \matrixel{\psi_{i}^{(0)}}{\op{W}}{\psi_{j}^{(0)}}.
                \end{equation*}

                Entonces,

                \begin{equation}
                    \op{W} = \matrice{W_{11}, W_{12}, W_{13}, W_{14}, W_{21}, W_{22}, W_{23}, W_{24}, W_{31}, W_{32}, W_{33}, W_{34}, W_{41}, W_{42}, W_{43}, W_{44}}.
                    \label{eq:matrixW}
                \end{equation}

                Y para simplificar la notación consideramos lo siguiente:

                \begin{align*}
                    \ket{1} &= \ket{200} = \dfrac{1}{\sqrt{2\pi a_{\mu}}}\dfrac{1}{2a_{\mu}}\left(1 - \dfrac{r}{2a_{\mu}}\right)\e^{-r/2a_{\mu}},\\
                    \ket{2} &= \ket{210} = \dfrac{1}{\sqrt{2\pi a_{\mu}}}\dfrac{1}{4a_{\mu}}r\e^{-r/2a_{\mu}}\cos\theta,\\
                    \ket{3} &= \ket{211} = \dfrac{1}{\sqrt{\pi a_{\mu}}}\dfrac{1}{8a_{\mu}^{2}}r\e^{-r/2a_{\mu}}\sin\theta\e^{\i\phi},\\
                    \ket{4} &= \ket{21-1} = \dfrac{1}{\sqrt{\pi a_{\mu}}}\dfrac{1}{8a_{\mu}^{2}}r\e^{-r/2a_{\mu}}\sin\theta\e^{-\i\phi}.
                \end{align*}

                Calculando cada uno de los elementos de matriz de \cref{eq:matrixW} donde únicamente consideramos la parte angular para determinar si el elemento de matriz se anula o no; tal que

                \begin{align*}
                    \dmatrixel{1}{W} &= \int(\ldots)\int_{0}^{\pi}\cos\theta\sin\theta\odif{\theta} = 0,\\
                    \matrixel{1}{W}{3} &= \matrixel{1}{W}{4} = \matrixel{3}{W}{1} = \matrixel{4}{W}{1} = \int(\ldots)\int_{0}^{\pi}\cos\theta\sin^{2}\theta\odif{\theta} = 0,\\
                    \dmatrixel{2}{W} &= \int(\ldots)\int_{0}^{\pi}\cos^{3}\theta\sin\theta\odif{\theta} = 0,\\
                    \matrixel{2}{W}{3} &= \matrixel{4}{W}{2} = \int(\ldots)\int_{0}^{2\pi}e^{\i\phi}\odif{\phi} = 0,\\
                    \matrixel{2}{W}{4} &= \matrixel{3}{W}{2} = \int(\ldots)\int_{0}^{2\pi}e^{-\i\phi}\odif{\phi} = 0,\\
                    \dmatrixel{3}{W} &= \dmatrixel{4}{W} = \matrixel{3}{W}{4} = \matrixel{4}{W}{3} = \int(\ldots)\int_{0}^{\pi}\cos\theta\sin^{3}\theta\odif{\theta} = 0,\\
                    \matrixel{1}{W}{2} &= \matrixel{2}{W}{1} \neq 0.
                \end{align*}

                Notamos que únicamente dos elementos de matriz son diferentes de cero, por lo que \cref{eq:matrixW} queda como; recordando el orden de la base es \(\ket{200}, \ket{210}, \ket{211}, \ket{21-1}\).

                \begin{equation*}
                    \op{W} = \matrice{0, \matrixel{1}{W}{2}, 0, 0, \matrixel{2}{W}{1}, 0, 0, 0, 0, 0, 0, 0, 0, 0, 0, 0}.
                \end{equation*}

                Dividiendo en cuadrantes,
                
                \begin{equation*}
                    \op{W} = 
                    \begin{pNiceArray}{lc:cr}
                        0 & \matrixel{1}{W}{2} & 0 & 0\\
                        \matrixel{2}{W}{1} & 0 & 0 & 0\\
                        \hdottedline
                        0 & 0 & 0 & 0\\
                        0 & 0 & 0 & 0
                    \end{pNiceArray}
                \end{equation*}

                Entonces la matriz asociada a la perturbación es

                \begin{equation*}
                    \op{W} = \matrice{0, \matrixel{1}{W}{2}, \matrixel{2}{W}{1}, 0}.
                    \label{eq:perturbationMatrixDiagonal}
                \end{equation*}

                Calculamos ahora los elementos de matriz notando que \(\matrixel{1}{W}{2} = \matrixel{2}{W}{1}^{*}\). Así,

                \begin{align}
                    \matrixel{1}{W}{2} &= \dfrac{1}{2\pi a_{\mu}}\dfrac{-\e E_{S}}{8a_{\mu}^{3}}\int_{0}^{\infty}\left(1 - \dfrac{r}{2a_{\mu}}\right)r^{4}\e^{-r/a_{\mu}}\odif{r}\int_{0}^{\pi}\cos^{2}\theta\sin\theta\odif{\theta}\int_{0}^{2\pi}\odif{\phi},\nonumber\\
                    &= 3a_{\mu}E_{S}\e,\\
                    \Aboxedsec{\matrixel{1}{W}{2} &= 3a_{\mu}E_{S}\e.}\label{eq:matrixElement1W2}
                \end{align}

                Sustituyendo \cref{eq:matrixElement1W2} en \cref{eq:perturbationMatrixDiagonal},

                \begin{equation*}
                    \op{W} = 3a_{\mu}E_{S}\e\matrice{0, 1, 1, 0}.
                \end{equation*}

                Notemos que la matriz anterior tiene la forma de \(\sigma_{x}\), cuyos eigenvalores conocemos, \idest \(\pm 1\) y sus eigenkets correspondientes son

                \begin{equation}
                    \ket{\psi_{+}} = \matrice{1, 1}\quad \text{y}\quad \ket{\psi_{-}} = \matrice{1, -1}.
                    \label{eq:eigenketsSigmaX}
                \end{equation}

                Por lo que las correcciones a primer orden de las energía son

                \begin{empheq}[box = \color{pinkwave}\fbox]{equation*}
                    E_{\pm}^{(1)} = \pm 3a_{\mu}E_{S}\e.
                \end{empheq}

                Y de \cref{eq:eigenketsSigmaX}, la nueva base está dada por

                \begin{empheq}[box = \color{pinkwave}\fbox]{align*}
                    \psi_{+} = \dfrac{1}{\sqrt{2}}(\ket{1} + \ket{2}),\\
                    \psi_{-} = \dfrac{1}{\sqrt{2}}(\ket{1} - \ket{2}).
                \end{empheq}

                ¿En cuántos niveles se desdobla la energía \(E_{2}\)? Se desdobla en 3 niveles.

                \begin{align*}
                    \psi_{+} &= \dfrac{1}{\sqrt{2}}(\ket{1} + \ket{2})\; \text{con energía}\quad E_{2} \simeq E_{0}^{(1)} + 3a_{\mu}E_{S}\e,\\
                    \psi_{-} &= \dfrac{1}{\sqrt{2}}(\ket{1} - \ket{2})\; \text{con energía}\quad E_{2} \simeq E_{0}^{(1)} - 3a_{\mu}E_{S}\e,\\
                    \ket{3} &= \ket{4}\; \text{con energía imperturbada}.
                \end{align*}
            \end{solution}
            
            \color{blue}
            \item Inciso extra (+2.5pt) - ¿Cuál es la mejor base para realizar el cálculo del inciso (b)? Encuentra el valor esperado del operador de momento dipolar eléctrico (\(\vect{\mathcal{P}} = -\e\vect{r}\)) para cada uno de los miembros de dicha base que son pertinentes para este problema.
            
            \textbf{Pista}: La ``mejor base'' es justamente aquella que diagonaliza a la matriz \(\op{W}{S}\).

            \color{black}
            \begin{solution}
                Calculamos únicamente el valor esperado de \(\vect{\mathcal{P}}\) para \(\psi_{+}\) y \(\psi_{-}\), ya que para \(\ket{2, 1, m = \pm 1}\) es 0.

                Así,

                \begin{equation*}
                    \dmatrixel{\psi_{\pm}}{\vect{\mathcal{P}}} = \int\left[\dfrac{1}{2}\left(\dfrac{\e^{-r/2a_{\mu}}}{2\pi a_{\mu}}\right)\left\lbrace\left(1 - \dfrac{r}{2a_{\mu}}\right) \pm \dfrac{1}{4a_{\mu}^{2}}r\cos\theta\right\rbrace\right]^{2}(-\e r)(\sin\theta\cos\phi\uveci + \sin\theta\sin\phi\uvecj + \cos\theta\uvectk)r^{2}\sin\theta\odif{r,\theta,\phi}.
                \end{equation*}

                Calculamos la integral para \(\phi\), de la cual únicamente sobrevive aquella en la dirección \(\uvectk\)

                \begin{align*}
                    \dmatrixel{\psi_{\pm}}{\vect{\mathcal{P}}} &= \mp \dfrac{\e\uvectk}{2a_{\mu}}\left(\dfrac{2}{3}\right)\int_{0}^{\infty}\dfrac{1}{8a_{\mu}^{3}}\left(1 - \dfrac{r}{2a_{\mu}}\right)r^{4}\e^{-r/a_{\mu}}\odif{r},\\
                    &= \mp \dfrac{\e\uvectk}{2a_{\mu}}\left(\dfrac{2}{3}\right)\left(\dfrac{2}{8a_{\mu}^{3}}\right)(-36a_{\mu}^{5}),\\
                    \Aboxedmain{\dmatrixel{\psi_{\pm}}{\vect{\mathcal{P}}} &= \pm 3a_{\mu}\e\uvectk.}
                \end{align*}
            \end{solution}
        \end{enumerate}
    \end{exercise}
\end{document}