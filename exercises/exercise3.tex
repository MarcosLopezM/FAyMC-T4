\documentclass[./../main.tex]{subfiles}
\graphicspath{{img/}}

\begin{document}
    \begin{exercise}[Estructura fina: el término de Darwin (valor total: 3pt)]
        Considera el átomo de un electrón. En clase se mostró que una de las correcciones relativistas está asociada a la incertidumbre en la posición del electrón. El término que describe esta corrección, conocido como término de Darwin, está dado por

        \begin{equation*}
            \observable{H}[3] = \dfrac{\pi\hbar^{2}}{2m^{2}c^{2}}\dfrac{Z\e^{2}}{4\pi\epsilon_{0}}\delta(r).
        \end{equation*}

        Utilizando teoría de perturbaciones independientes del tiempo demuestra que, a primer orden, la corrección de la energía debida al término de Darwin es

        \begin{equation*}
            \change{E_{3}} =
            \begin{cases*}
                0 & si \(\ell \neq 0\)\\
                -E_{n}\tfrac{Z^{2}\alpha^{2}}{n} & si \(\ell = 0\)
            \end{cases*}
        \end{equation*}

        en donde \(E_{n} = -\mu c^{2}\alpha^{2}Z^{2}/(2n^{2})\) son las energías imperturbadas del sistema.

        \textbf{Sugerencia:} A pesar de que el átomo de un electrón es un sistema degenerado, este problema puede resolverse usando el caso no degenerado de la teoría de perturbaciones independientes del tiempo, ¿por qué? ¿cómo podrías demostrarlo?
    \end{exercise}
\end{document}