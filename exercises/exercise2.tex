\documentclass[./../main.tex]{subfiles}
\graphicspath{{img/}}

\begin{document}
    \begin{exercise}[Átomo de hidrógeno en un estado mezclado]
        El electrón de un átomo de hidrógeno se encuentra en el siguiente estado de posición y spin:

        \begin{equation*}
            \psi(r, \theta, \phi) = R_{2,1}(r) \left(\sqrt{\tfrac{1}{3}}Y_{1,0}(\theta, \phi)\chi_{+} + \sqrt{\tfrac{2}{3}}Y_{1,1}(\theta, \phi)\chi_{-}\right),
        \end{equation*}

        en donde las funciones \(\chi_{+}\) y \(\chi_{-}\) corresponden a las funciones de spin del electrón \(\ket{+} \equiv \ket{s = 1/2, m_{s} = 1/2}\) y \(\ket{-} \equiv \ket{s = 1/2, m_{s} = -1/2}\), respectivamente.

        \begin{enumerate}
            \item Valor: 0.5pt - Si se realiza una medida del momento angular orbital total de \(\observable{L}[][2]\), ¿cuáles son los posibles resultados y cuál es la probabilidad de obtener cada uno de ellos?
            \item Valor: 0.5pt - Misma pregunta para la componente-\(z\) del momento angular orbital \(\observable{L}[z]\).
            \item Valor: 0.5pt - Misma pregunta para la componente-\(z\) del momento angular de spin total \(\observable{S}[][2]\).
            \item Valor: 0.5pt - Misma pregunta para la componente-\(z\) del operador de spin, \(\observable{S}[z]\).
            \item Valor: 0.5pt - Si se mide la posición de la partícula, ¿cuál es la \textbf{densidad de probabilidad} de encontrarla en la posición \((r, \theta, \phi)\)?
            \item Valor: 0.5pt - Ahora se miden tanto la componente-\(z\) del spin como la distancia \(r\) al origen. Nota que estas cantidades físicas pueden medirse simultáneamente. Demuestra que la \textbf{densidad de probabilidad} de encontrar a la partícula en el estado de spin \(\chi_{+}\) a una distancia \(r\) del origen está dada por
            
            \begin{equation*}
                \tfrac{1}{72a_{0}^{5}}r^{2}\e^{-r/a_{0}},
            \end{equation*}

            en donde \(a_{0}\) es el radio de Bohr.

            \color{blue}
            \item Extra: Valor +0.5pt - Sea \(\observable{\vect{J}} = \observable{\vect{L}} + \observable{\vect{S}}\) el momento angular total del sistema. Si se mide \(\observable{J}[][2]\), ¿cuáles son los posibles resultados y cuál es la probabilidad de obtener cada uno de ellos?
            \item Extra: Valor +0.5pt - Misma pregunta para \(\observable{J}[z]\).
            
            \textbf{Sugerencia}: Para resolver los incisos (g) y (h) puedes usar (sin demostrar) los resultados que encontraste en la Tarea 1 para la suma de momentos angulares \(j_{1} = 1\) y \(j_{2} = 1/2\). 
        \end{enumerate}
    \end{exercise}
\end{document}