\documentclass[./../main.tex]{subfiles}
\graphicspath{{img/}}
\begin{document}
	\begin{exercise}
		¿Podrían existir hadrones compuestos por una combinación cuark-anticuark-cuark-anticuark, o mesones formados por puros gluones? Desarrolla tu respuesta.

		\begin{solution}
			La respuesta corta es sí, ya que la teoría nos dice que no hay un límite de cuarks que formen un hadron, siempre y cuando su carga de color sea neutra, \idest que exista el mismo número de cuarks y anticuarks.

			Para el caso de los hadrones, estas combinaciones se conocen como tetracuarks, que son mesones exóticos, cuya configuración tiene la forma \(q\overline{q}Q\overline{Q}\), donde \(q\) representa cuarks ligeros (\emph{up}, \emph{down} o \emph{strange}) y \(Q\) cuarks pesados (\emph{charm} o \emph{bottom}).

			Algunos candidatos a tetracuarks son:

			\begin{itemize}[twocol]
				\item X(3872)
				\item X(4274)
				\item X(4500)
				\item X(4700)
				\item Z(4430)
				\item Y(4140)
				\item \(Z_{c}\)(3900)
			\end{itemize}

			Estos hadrones de cuatro cuarks configurados de esta manera forman parte de los llamados mesones exóticos, así como los mesones formados por puros gluones, que se conocen como \emph{gluonballs} o \emph{gluonium}, que son partículas posibles de existir debido a la respuesta dada en el problema anterior.
		\end{solution}
	\end{exercise}
\end{document}
