\documentclass[./../main.tex]{subfiles}
\graphicspath{{img/}}
\begin{document}
	\begin{exercise}
		¿Por qué los gluones son los únicos bosones de norma que pueden interactuar entre sí? ¿Puede un gluon provocar el cambio de carga de color en un cuark?

		\begin{solution}
			En efecto, los gluones son los únicos bosones de norma que pueden interactuar entre sí porque a diferencia de los otros bosones de norma, estos tienen una carga de color. Por lo tanto, los gluones pueden interactuar entre sí mediante la interacción fuerte, ya que además son portadores de esta. Y de esta interacción surge la teoría de la cromodinámica cuántica (QCD, por sus siglas en inglés).

			Entonces, ¿los gluones pueden cambiar la carga de color de un cuark? La respuesta es que sí, ya que al ser los gluones mediadores y portadores de la fuerza nuclear fuerte, si un gluon con cierta carga de color interactúa, ya sea que un cuark emita o absorba un gluon, este cambiará su carga de color.
		\end{solution}
	\end{exercise}
\end{document}
